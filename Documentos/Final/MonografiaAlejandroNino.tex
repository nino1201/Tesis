\documentclass[12pt]{article}
\usepackage[utf8]{inputenc}
\usepackage[espanol]{babel}
\usepackage{polyglossia}


\usepackage{graphicx}
\usepackage{braket}
\graphicspath{{imagenes/}}
\usepackage{amsmath}
\usepackage{enumerate}
\usepackage{ dsfont }
\usepackage{ mathrsfs }
\usepackage{bbold}
\usepackage{ amssymb }
\usepackage{dsfont}
\usepackage{ mathrsfs }
\usepackage{centernot}
\usepackage{mathtools}
\usepackage{ stmaryrd }
\usepackage{physics}
\usepackage{titlesec}
\usepackage{geometry}
\usepackage{mathtools}
\usepackage{bbold}
\usepackage{caption}
\usepackage{cancel}
\usepackage{subcaption}
\renewcommand{\baselinestretch}{1.6}
\geometry{
 letterpaper,
 left=28mm, right=28mm,
 top=27mm,bottom=27mm
 }


\begin{document}

\begin{center}
\includegraphics[width=2cm]{logo.png}\\
\vspace{1cm}
\textbf{{\LARGE Estudio de Monte Carlo de la eficiencia
cuántica de diferentes dectores de rayos-X }}\\
\vspace{2.3cm}
{\LARGE{Proyecto de grado como requisito parcial para optar el título de }}\\
\vspace{1.0cm}
\centering
{\LARGE{Físico}}\\
\vspace{2.0cm}
{\large Alejandro Niño Chaparro}\\
\vspace{2.3cm}
Universidad de los Andes\\
Facultad de Ciencias\\
Departamento de Física\\
\vspace{2.3cm}
Bogotá D.C.\hspace{0.1cm}Colombia\\
2018
\newpage
\thispagestyle{empty}
\includegraphics[width=2cm]{logo.png}\\
\vspace{1cm}
\textbf{{\LARGE Estudio de Monte Carlo de la eficiencia
cuántica de diferentes dectores de rayos-X}}\\
\vspace{1.3cm}
{\LARGE{Proyecto de grado como requisito parcial para optar el título de}}\\
\vspace{1.0cm}
\centering
{\LARGE{Físico}}\\
\vspace{2.0cm}
{\large Alejandro Niño Chaparro}\\
\vspace{2.3cm}
Universidad de los Andes\\
Facultad de Ciencias\\
Departamento de Física\\
\vspace{1.3cm}
Director\\
\textbf{\large{Carlos Arturo Ávila Bernal. Ph.D}}\\
\vspace{0.5cm}
Bogotá D.C.\hspace{0.1cm}Colombia\\
2018
\end{center}
\tableofcontents % indice de contenidos
\listoftables % indice de tablas


\end{document}
